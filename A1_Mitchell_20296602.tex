% Options for packages loaded elsewhere
\PassOptionsToPackage{unicode}{hyperref}
\PassOptionsToPackage{hyphens}{url}
%
\documentclass[
]{article}
\usepackage{amsmath,amssymb}
\usepackage{iftex}
\ifPDFTeX
  \usepackage[T1]{fontenc}
  \usepackage[utf8]{inputenc}
  \usepackage{textcomp} % provide euro and other symbols
\else % if luatex or xetex
  \usepackage{unicode-math} % this also loads fontspec
  \defaultfontfeatures{Scale=MatchLowercase}
  \defaultfontfeatures[\rmfamily]{Ligatures=TeX,Scale=1}
\fi
\usepackage{lmodern}
\ifPDFTeX\else
  % xetex/luatex font selection
\fi
% Use upquote if available, for straight quotes in verbatim environments
\IfFileExists{upquote.sty}{\usepackage{upquote}}{}
\IfFileExists{microtype.sty}{% use microtype if available
  \usepackage[]{microtype}
  \UseMicrotypeSet[protrusion]{basicmath} % disable protrusion for tt fonts
}{}
\makeatletter
\@ifundefined{KOMAClassName}{% if non-KOMA class
  \IfFileExists{parskip.sty}{%
    \usepackage{parskip}
  }{% else
    \setlength{\parindent}{0pt}
    \setlength{\parskip}{6pt plus 2pt minus 1pt}}
}{% if KOMA class
  \KOMAoptions{parskip=half}}
\makeatother
\usepackage{xcolor}
\usepackage[margin=1in]{geometry}
\usepackage{color}
\usepackage{fancyvrb}
\newcommand{\VerbBar}{|}
\newcommand{\VERB}{\Verb[commandchars=\\\{\}]}
\DefineVerbatimEnvironment{Highlighting}{Verbatim}{commandchars=\\\{\}}
% Add ',fontsize=\small' for more characters per line
\usepackage{framed}
\definecolor{shadecolor}{RGB}{248,248,248}
\newenvironment{Shaded}{\begin{snugshade}}{\end{snugshade}}
\newcommand{\AlertTok}[1]{\textcolor[rgb]{0.94,0.16,0.16}{#1}}
\newcommand{\AnnotationTok}[1]{\textcolor[rgb]{0.56,0.35,0.01}{\textbf{\textit{#1}}}}
\newcommand{\AttributeTok}[1]{\textcolor[rgb]{0.13,0.29,0.53}{#1}}
\newcommand{\BaseNTok}[1]{\textcolor[rgb]{0.00,0.00,0.81}{#1}}
\newcommand{\BuiltInTok}[1]{#1}
\newcommand{\CharTok}[1]{\textcolor[rgb]{0.31,0.60,0.02}{#1}}
\newcommand{\CommentTok}[1]{\textcolor[rgb]{0.56,0.35,0.01}{\textit{#1}}}
\newcommand{\CommentVarTok}[1]{\textcolor[rgb]{0.56,0.35,0.01}{\textbf{\textit{#1}}}}
\newcommand{\ConstantTok}[1]{\textcolor[rgb]{0.56,0.35,0.01}{#1}}
\newcommand{\ControlFlowTok}[1]{\textcolor[rgb]{0.13,0.29,0.53}{\textbf{#1}}}
\newcommand{\DataTypeTok}[1]{\textcolor[rgb]{0.13,0.29,0.53}{#1}}
\newcommand{\DecValTok}[1]{\textcolor[rgb]{0.00,0.00,0.81}{#1}}
\newcommand{\DocumentationTok}[1]{\textcolor[rgb]{0.56,0.35,0.01}{\textbf{\textit{#1}}}}
\newcommand{\ErrorTok}[1]{\textcolor[rgb]{0.64,0.00,0.00}{\textbf{#1}}}
\newcommand{\ExtensionTok}[1]{#1}
\newcommand{\FloatTok}[1]{\textcolor[rgb]{0.00,0.00,0.81}{#1}}
\newcommand{\FunctionTok}[1]{\textcolor[rgb]{0.13,0.29,0.53}{\textbf{#1}}}
\newcommand{\ImportTok}[1]{#1}
\newcommand{\InformationTok}[1]{\textcolor[rgb]{0.56,0.35,0.01}{\textbf{\textit{#1}}}}
\newcommand{\KeywordTok}[1]{\textcolor[rgb]{0.13,0.29,0.53}{\textbf{#1}}}
\newcommand{\NormalTok}[1]{#1}
\newcommand{\OperatorTok}[1]{\textcolor[rgb]{0.81,0.36,0.00}{\textbf{#1}}}
\newcommand{\OtherTok}[1]{\textcolor[rgb]{0.56,0.35,0.01}{#1}}
\newcommand{\PreprocessorTok}[1]{\textcolor[rgb]{0.56,0.35,0.01}{\textit{#1}}}
\newcommand{\RegionMarkerTok}[1]{#1}
\newcommand{\SpecialCharTok}[1]{\textcolor[rgb]{0.81,0.36,0.00}{\textbf{#1}}}
\newcommand{\SpecialStringTok}[1]{\textcolor[rgb]{0.31,0.60,0.02}{#1}}
\newcommand{\StringTok}[1]{\textcolor[rgb]{0.31,0.60,0.02}{#1}}
\newcommand{\VariableTok}[1]{\textcolor[rgb]{0.00,0.00,0.00}{#1}}
\newcommand{\VerbatimStringTok}[1]{\textcolor[rgb]{0.31,0.60,0.02}{#1}}
\newcommand{\WarningTok}[1]{\textcolor[rgb]{0.56,0.35,0.01}{\textbf{\textit{#1}}}}
\usepackage{graphicx}
\makeatletter
\def\maxwidth{\ifdim\Gin@nat@width>\linewidth\linewidth\else\Gin@nat@width\fi}
\def\maxheight{\ifdim\Gin@nat@height>\textheight\textheight\else\Gin@nat@height\fi}
\makeatother
% Scale images if necessary, so that they will not overflow the page
% margins by default, and it is still possible to overwrite the defaults
% using explicit options in \includegraphics[width, height, ...]{}
\setkeys{Gin}{width=\maxwidth,height=\maxheight,keepaspectratio}
% Set default figure placement to htbp
\makeatletter
\def\fps@figure{htbp}
\makeatother
\setlength{\emergencystretch}{3em} % prevent overfull lines
\providecommand{\tightlist}{%
  \setlength{\itemsep}{0pt}\setlength{\parskip}{0pt}}
\setcounter{secnumdepth}{-\maxdimen} % remove section numbering
\ifLuaTeX
  \usepackage{selnolig}  % disable illegal ligatures
\fi
\usepackage{bookmark}
\IfFileExists{xurl.sty}{\usepackage{xurl}}{} % add URL line breaks if available
\urlstyle{same}
\hypersetup{
  pdftitle={Shark Fin Volume Analysis Report},
  pdfauthor={Emma Mitchell},
  hidelinks,
  pdfcreator={LaTeX via pandoc}}

\title{Shark Fin Volume Analysis Report}
\author{Emma Mitchell}
\date{2025-01-07}

\begin{document}
\maketitle

\subsection{\texorpdfstring{\href{https://github.com/emmaamitchyy/biol432}{GitHub
Repository}}{GitHub Repository}}\label{github-repository}

\subsubsection{Introduction}\label{introduction}

This report analyzes shark fin measurement data, including fin width and
length, and estimates the fin volume for various shark species using R.
The shark species include Great White Sharks, Bull Sharks, Whale Sharks,
Black tip Sharks, and Galapagos Sharks. The data is processed, analyzed
and visualized using various R scripts. The goal of this report is to
gain insights into fin volumes across shark species.

\subsubsection{Data Generation:}\label{data-generation}

Load the Needed Packages:

\begin{Shaded}
\begin{Highlighting}[]
\FunctionTok{library}\NormalTok{(dplyr)}
\end{Highlighting}
\end{Shaded}

\begin{verbatim}
## 
## Attaching package: 'dplyr'
\end{verbatim}

\begin{verbatim}
## The following objects are masked from 'package:stats':
## 
##     filter, lag
\end{verbatim}

\begin{verbatim}
## The following objects are masked from 'package:base':
## 
##     intersect, setdiff, setequal, union
\end{verbatim}

\begin{Shaded}
\begin{Highlighting}[]
\FunctionTok{library}\NormalTok{(tidyverse)}
\end{Highlighting}
\end{Shaded}

\begin{verbatim}
## -- Attaching core tidyverse packages ------------------------ tidyverse 2.0.0 --
## v forcats   1.0.0     v readr     2.1.5
## v ggplot2   3.5.1     v stringr   1.5.1
## v lubridate 1.9.4     v tibble    3.2.1
## v purrr     1.0.2     v tidyr     1.3.1
## -- Conflicts ------------------------------------------ tidyverse_conflicts() --
## x dplyr::filter() masks stats::filter()
## x dplyr::lag()    masks stats::lag()
## i Use the conflicted package (<http://conflicted.r-lib.org/>) to force all conflicts to become errors
\end{verbatim}

\begin{Shaded}
\begin{Highlighting}[]
\FunctionTok{library}\NormalTok{(ggplot2)}
\end{Highlighting}
\end{Shaded}

\begin{Shaded}
\begin{Highlighting}[]
\FunctionTok{library}\NormalTok{(tidyr)}
\end{Highlighting}
\end{Shaded}

Before running the analysis, the data is generated using the
`data\_generator.R' script. This script creates the `measurement.csv'
file, which contains data about shark species and fin measurements. This
data set is then used to estimate fin volumes and perform further
analysis

\begin{Shaded}
\begin{Highlighting}[]
\FunctionTok{source}\NormalTok{(}\StringTok{\textquotesingle{}datagenerato.R\textquotesingle{}}\NormalTok{)}
\end{Highlighting}
\end{Shaded}

The generated data is then used to estimate the volume of various shark
species fins. This value is then used to perform further analysis and
compare fins across the five species measured.

\begin{Shaded}
\begin{Highlighting}[]
\FunctionTok{source}\NormalTok{(}\StringTok{\textquotesingle{}volumeEstimato.R\textquotesingle{}}\NormalTok{)}
\end{Highlighting}
\end{Shaded}

In order to run analyses, open the measurement data generated from the
volume estimator scrip:

\begin{Shaded}
\begin{Highlighting}[]
\NormalTok{data}\OtherTok{\textless{}{-}}\FunctionTok{read.csv}\NormalTok{(}\StringTok{"measurements.csv"}\NormalTok{)}
\end{Highlighting}
\end{Shaded}

\begin{Shaded}
\begin{Highlighting}[]
\FunctionTok{head}\NormalTok{(data)}
\end{Highlighting}
\end{Shaded}

\begin{verbatim}
##          Species  Fin_Width Limb_Length Observer       Volume
## 1    whale_shark  0.1107716    178.3582   Olivia    0.5729528
## 2    whale_shark  1.2795166    179.4478   Olivia   76.9127921
## 3     bull_shark  4.5308327    204.6311   Olivia 1099.7557579
## 4     bull_shark  6.6261812    126.9494 Meredith 1459.2376975
## 5    whale_shark 17.9230591    117.8107      Ben 9907.8099131
## 6 blacktip_shark  0.3644350    128.5974      Ben    4.4713723
\end{verbatim}

\subsubsection{Data Analysis:}\label{data-analysis}

To ensure clarity and organization in the data set, the data is sorted
by species, observer, and then limb volume. This sorting allows us to
easily group and compare measurements within each species and observer
combination, while also identifying trends or outliers in limb volume.
Organizing the data in this manner provides a structured foundation for
further analysis and visualization.

\begin{Shaded}
\begin{Highlighting}[]
\NormalTok{sorted\_data }\OtherTok{\textless{}{-}}\NormalTok{ data }\SpecialCharTok{\%\textgreater{}\%} \FunctionTok{arrange}\NormalTok{(Species, Observer, Volume)}
\end{Highlighting}
\end{Shaded}

\begin{Shaded}
\begin{Highlighting}[]
\FunctionTok{head}\NormalTok{(sorted\_data)}
\end{Highlighting}
\end{Shaded}

\begin{verbatim}
##          Species Fin_Width Limb_Length Observer       Volume
## 1 blacktip_shark  0.364435   128.59737      Ben     4.471372
## 2 blacktip_shark 10.359171    78.71772      Ben  2211.521092
## 3 blacktip_shark 13.538955   110.41288      Ben  5298.569416
## 4 blacktip_shark 10.676751   258.32673      Ben  7709.322801
## 5 blacktip_shark 16.849306   143.56846      Ben 10670.670851
## 6 blacktip_shark 18.482903   136.34827      Ben 12194.349591
\end{verbatim}

The following table provides an overview of the average volume for five
well-known shark species, showcasing the substantial differences in size
between them.

\begin{Shaded}
\begin{Highlighting}[]
\NormalTok{avg\_vol }\OtherTok{\textless{}{-}}\NormalTok{ sorted\_data }\SpecialCharTok{\%\textgreater{}\%}
  \FunctionTok{group\_by}\NormalTok{(Species) }\SpecialCharTok{\%\textgreater{}\%}
  \FunctionTok{summarise}\NormalTok{(}\AttributeTok{Average\_Volume =} \FunctionTok{mean}\NormalTok{(Volume))}

\CommentTok{\#display table: }
\NormalTok{avg\_vol}
\end{Highlighting}
\end{Shaded}

\begin{verbatim}
## # A tibble: 5 x 2
##   Species           Average_Volume
##   <chr>                      <dbl>
## 1 blacktip_shark             4723.
## 2 bull_shark                 5582.
## 3 galapagos_shark            4286.
## 4 great_white_shark          5524.
## 5 whale_shark                5506.
\end{verbatim}

These variations highlight the diversity in size and mass across
different shark species, which can be influenced by factors like age,
environmental conditions, and genetic traits.

The table below illustrates the number of observations recorded for each
shark species by different observers. This data helps to provide a
clearer picture of how often each species is encountered and studied,
reflecting both the frequency of sightings and the efforts of individual
observers.

\begin{Shaded}
\begin{Highlighting}[]
\CommentTok{\# Calculate the number of observations for each species{-}observer combination: }

\NormalTok{obs\_count }\OtherTok{\textless{}{-}}\NormalTok{ sorted\_data }\SpecialCharTok{\%\textgreater{}\%} 
  \FunctionTok{group\_by}\NormalTok{(Species, Observer) }\SpecialCharTok{\%\textgreater{}\%}
  \FunctionTok{summarise}\NormalTok{(}\AttributeTok{num\_observations =} \FunctionTok{n}\NormalTok{())}
\end{Highlighting}
\end{Shaded}

\begin{verbatim}
## `summarise()` has grouped output by 'Species'. You can override using the
## `.groups` argument.
\end{verbatim}

\begin{Shaded}
\begin{Highlighting}[]
\CommentTok{\#display the table of observations: }
\NormalTok{obs\_count}
\end{Highlighting}
\end{Shaded}

\begin{verbatim}
## # A tibble: 15 x 3
## # Groups:   Species [5]
##    Species           Observer num_observations
##    <chr>             <chr>               <int>
##  1 blacktip_shark    Ben                     7
##  2 blacktip_shark    Meredith                3
##  3 blacktip_shark    Olivia                  9
##  4 bull_shark        Ben                     8
##  5 bull_shark        Meredith                6
##  6 bull_shark        Olivia                  6
##  7 galapagos_shark   Ben                     2
##  8 galapagos_shark   Meredith                5
##  9 galapagos_shark   Olivia                 10
## 10 great_white_shark Ben                     9
## 11 great_white_shark Meredith                4
## 12 great_white_shark Olivia                  8
## 13 whale_shark       Ben                     9
## 14 whale_shark       Meredith                5
## 15 whale_shark       Olivia                  9
\end{verbatim}

\subsubsection{Results:}\label{results}

The following box plot is important for visualizing the distribution of
fin volumes across different shark species. This plot highlights the
variation in fin sizes, showing the median, quartiles, and potential
outliers for each species. It provides a clear comparison of fin volume
distributions, helping to illustrate the differences in size between
species such as the Great White Shark, Bull Shark, Whale Shark,
Galapagos Shark, and Black tip Shark.

\begin{Shaded}
\begin{Highlighting}[]
\CommentTok{\#box plot to compare the distributions of fin volumes for each species of shark:}
\FunctionTok{ggplot}\NormalTok{(sorted\_data, }\FunctionTok{aes}\NormalTok{(}\AttributeTok{x=}\NormalTok{Species, }\AttributeTok{y=}\NormalTok{Volume))}\SpecialCharTok{+}
  \FunctionTok{geom\_boxplot}\NormalTok{()}\SpecialCharTok{+}
  \FunctionTok{labs}\NormalTok{(}\AttributeTok{title =} \StringTok{"Box Plot of Fin Volume by Species of Shark"}\NormalTok{,}
       \AttributeTok{x =} \StringTok{"Shark Species"}\NormalTok{, }\AttributeTok{y =} \StringTok{"Fin Volume(cm\^{}3)"}\NormalTok{)}\SpecialCharTok{+}
  \FunctionTok{theme}\NormalTok{(}\AttributeTok{axis.text.x =} \FunctionTok{element\_text}\NormalTok{(}\AttributeTok{angle =} \DecValTok{45}\NormalTok{, }\AttributeTok{hjust =} \DecValTok{1}\NormalTok{))}
\end{Highlighting}
\end{Shaded}

\includegraphics{A1_Mitchell_20296602_files/figure-latex/unnamed-chunk-13-1.pdf}

\textbf{Figure 1. Box plot of fin volumes by shark species}\emph{.}

This plot displays the distribution of fin volumes for five shark
species: Great White Shark, Bull Shark, Whale Shark, Galapagos Shark,
and Black tip Shark. The boxes represent the interquartile range (IQR),
with the median indicated by the line inside the box. Outliers are
marked as individual points, providing insights into the variation in
fin size within each species

A multi-panel plot was created to display the frequency histograms
showing the distribution of limb volumes for each shark species. This
visualization provides an effective way to compare the variations in
limb volumes across different species. To facilitate the creation of
this plot, the data was first transformed from a wide to a long format
using appropriate R code. This allowed for easier plotting of the data
by species. Each panel in the plot corresponds to a different species,
and the histograms illustrate the frequency distribution of limb volumes
within each species. This approach helps in understanding the
differences and patterns in limb volume across the species in the study.

\begin{Shaded}
\begin{Highlighting}[]
\CommentTok{\# Convert the data to long format}
\NormalTok{shark\_data\_long }\OtherTok{\textless{}{-}}\NormalTok{ sorted\_data }\SpecialCharTok{\%\textgreater{}\%}
  \FunctionTok{pivot\_longer}\NormalTok{(}\AttributeTok{cols =}\NormalTok{ Volume, }\AttributeTok{names\_to =} \StringTok{"Measurement"}\NormalTok{, }\AttributeTok{values\_to =} \StringTok{"Value"}\NormalTok{)}

\CommentTok{\# Create a multi{-}panel histogram of fin volume distributions for each species}
\FunctionTok{ggplot}\NormalTok{(shark\_data\_long, }\FunctionTok{aes}\NormalTok{(}\AttributeTok{x =}\NormalTok{ Value)) }\SpecialCharTok{+}
  \FunctionTok{geom\_histogram}\NormalTok{(}\AttributeTok{binwidth =} \DecValTok{10}\NormalTok{, }\AttributeTok{fill =} \StringTok{"blue"}\NormalTok{, }\AttributeTok{color =} \StringTok{"black"}\NormalTok{) }\SpecialCharTok{+}
  \FunctionTok{facet\_wrap}\NormalTok{(}\SpecialCharTok{\textasciitilde{}}\NormalTok{Species) }\SpecialCharTok{+}
  \FunctionTok{theme\_minimal}\NormalTok{() }\SpecialCharTok{+}
  \FunctionTok{labs}\NormalTok{(}\AttributeTok{title =} \StringTok{"Distribution of Fin Volume by Shark Species"}\NormalTok{, }\AttributeTok{x =} \StringTok{"Fin Volume (cm³)"}\NormalTok{, }\AttributeTok{y =} \StringTok{"Frequency"}\NormalTok{)}
\end{Highlighting}
\end{Shaded}

\includegraphics{A1_Mitchell_20296602_files/figure-latex/unnamed-chunk-14-1.pdf}

\textbf{Figure 2. Distribution of fin volume by shark species}\emph{.}

This multi-panel histogram shows the frequency distribution of fin
volumes (in cm³) for each shark species. The data was converted from
wide to long format using the pivot\_longer function, with the fin
volume measurements labeled as ``Value'' and grouped by species. Each
panel represents a different species of shark, and the bin width for the
histograms is set to 10 cm³. The plot highlights the variation in fin
volume across shark species, with each species' distribution displayed
separately for comparison.

\end{document}
